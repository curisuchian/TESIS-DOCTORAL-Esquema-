%Deberá contener el planteo del problema a resolver, los objetivos generales y específicos y una breve explicación de lo que versará la Tesis.

Los bosques nativos presentes en la Tierra brindan una serie de servicios ecosistémicos que benefician a la humanidad, como la captación y filtración de agua, la generación de oxígeno y asimilación de diversos contaminantes, la protección de la biodiversidad, retención de suelo, refugio de fauna silvestre, belleza escénica, entre otros \cite{}[referencias]. Así, tomar acciones tendientes a su preservación puede ser visto como una cuestión moral, ética, altruista, pero sobre todo, de sobrevivencia.
Está ampliamente probado el efecto que produce la liberación de carbono en el cambio climático global \cite{}[referencias]. La deforestación y la degradación forestal ocupan el segundo lugar en actividades que contribuyen con la emisión de gases de efecto invernadero, superados solamente por la utilización de combustibles fósiles [Simula, 2011]. Es decir, un bosque que se degrada, no solamente deja de absorber dióxido de carbono del aire, sino que también aporta fuertemente a su liberación.
Afortunadamente, en muchas partes del mundo están proliferando las reservas naturales. De forma específica, en la provincia de Misiones, Argentina, casi la mitad del territorio se encuentra protegido por algún tipo de legislación nacional, provincial o municipal. Así mismo, abundan las reservas privadas interesadas en aportar a la preservación global sí, pero también incentivados por un valor económico impulsado por variadas actividades turísticas. Todas estas acciones en pro de la preservación, demandan, como toda actividad consciente, conocer cuan bien se está haciendo el trabajo.
En 2.007, el Panel Internacional por el Cambio Climático (\textit{International Panel on Climate Change} - IPCC) [Climate Change 2007 - The Physical Science Basis: Working Group…] recomendó el uso de una combinación de datos de observación terrestre (\textit{Earth observation} - EO) e inventarios en campo para estimar el área forestal, el stock de carbono y posibles cambios.  Unos años más tarde, la Convención Marco de las Naciones Unidas sobre el Cambio Climático (\textit{United Nations Framework Convention on Climate Change} - UNFCCC) adoptó un mecanismo que provee incentivos financieros para la reducción de emisiones, llamado REDD+ \cite{pistorius_red_2012}. Para implementarlos, los países son apremiados a establecer un sistema nacional de medición, reporte y verificación (\textit{measurement, reporting and verification} - MRV). 
En cuanto a los cambios en el área forestal, el impacto de la degradación varía desde finos cambios a nivel estructural en altura y dosel, sutiles disrupciones en los servicios ecosistémicos a pérdidas de biomasa a gran escala. Estos cambios pueden ocurrir en variadas escalas espaciales y temporales. La floresta degradada puede asumir una cobertura de dosel similar a una floresta intacta, pero tiene menos biomasa, en algunos casos reducida hasta un 75\% \cite{change_report_2006}.Diferentes tipos de floresta responderán de manera diferente a los cambios, con tasas de recuperación variables, dependientes de la localización y tipo y de la intensidad y extensión de la degradación. Por ello, una sola estrategia de monitoreo no puede ser apropiada a una aplicación a gran escala. En su lugar, un procedimiento específico, adecuado a cada región es requerido \cite{mitchell_current_2017}. Además. deberían ser considerados aspectos culturales y económicos de cada región. Así, con ese sentido, esta tesis viene a aportar en diversos aspectos al desarrollo de un sistema de monitoreo de la selva misionera.
Cabe resaltar que actualmente Argentina cuenta con un sistema MRV. Este consiste en…
Sin embargo, la gran cantidad de publicaciones técnicas evidencian que puede ser mejorado a partir de estudios como la presente tesis…
