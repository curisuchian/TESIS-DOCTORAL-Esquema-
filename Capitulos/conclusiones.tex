%Deben presentarse en párrafos cortos y concretos. No deben hacer referencia a trabajos futuros ni a hipótesis no incluidas en el trabajo
\color{black} 
\section{CONCLUSIONES Bloque 1}
Como surge de los resultados del análisis comparativo entre las diferentes plataformas para captura de imágenes, los VANT se adaptan mejor a la realidad de las reservas de bosque nativo existentes en la Provincia de Misiones, que abarcan desde algunas ecenas de hectáreas hasta cientos o miles de hectáreas en extensión. Asimismo muchas de ellas no están conectadas entre sí, y son de diferentes propietarios, tanto gubernamentales como privados, con lo cual el uso de VANT como herramienta de relevamiento cobra mayor sentido. \color{magenta}(podríamos plasmar aquí lo que hemos conversado sobre el relevamiento muestral, en áreas cercanas a rutas y caminos, para conformar una especie de espina dorsal de pescado)
\color{black} 
\section{CONCLUSIONES homomórfico}
Los resultados de las pruebas indican que para el conteo de sombras deben considerarse varios factores. En principio la baja calidad de las imágenes utilizadas para las pruebas era evidente, y necesariamente debería hacerse un preprocesamiento de las imágenes, como ser ajustes y ecualizaciones de las componentes de color, para mejorar el contraste. Otra limitación para el uso de ésta técnica para el conteo de ejemplares de A. polyneuron es que debe reforzarse la contundencia de la afirmación de que esas sombras seleccionadas son efectivamente de dicha especie. Al no disponer de los datos asociados a las imágenes como ser fecha y hora exacta de la captura, se hacía difícil discernir si las sombras correspondían a árboles altos, o las sombras eran más intensas por el horario en el que fueron capturadas las imágenes (por ejemplo, en el crepúsculo) Hay que mencionar también las diferencias que hubo entre la detección de sombras por parte del algoritmo y la detección manual. Para el caso de la búsqueda automática con tamaño de ventana de 20 píxeles hubo casos de detección de falsos positivos, es decir, detectaba sombras donde el ojo humano no lo hacía, y también omitía seleccionar sombras que a simple vista serían consideradas como sombras de interés. Cabe añadir que la limitación de facto para el algoritmo es que realiza la selección de sombras con base en la observación de la componente intensidad del modelo HSI. Al ojo humano le resulta más fácil distinguir sombras en imágenes a color.

\subsection{Conclusiones IIC} \label{Conclusiones}

El procedimiento propuesto usando el índice invariante de color se puede aplicar para la detección de sombras en imágenes aéreas de áreas selváticas, usando recursos computacionales relativamente simples. Los resultados mostraron que el 85º percentil de la distribución de frecuencia del índice invariante de color obtenido por medio de la ecuación \ref{invariante de color} y la diferencia entre canales como el numerador, son parámetros aceptables para calcular el valor de umbral para la binarización y desde ahí obtener la máscara que corresponde a las áreas sombreadas.
En ambos casos, en que los numeradores son la diferencia entre azul y verde y azul y rojo, se presentaron los resultados mejores. Se encontró un método de comparación que permite evaluar el desempeño del algoritmo, en el que el índice de calidad QI se definió como el cociente entre la intersección resultante entre ambas máscaras, manual y automática, sobre la unión resultante de ambas. De esta manera, si el índice se aproxima al valor 1 implica un mayor nivel de coincidencia entre ambas máscaras. Los resultados del algoritmo se comparan con la selección manual llevada a cabo por expertos, como puede apreciarse de las máscaras superpuestas sobre las imágenes. Si bien este método tiene algunas limitaciones, se puede aplicar sobre imágenes RGB, las que tienen algunos condicionantes en el momento de captura, y condiciones meteorológicas, como la presencia de nubes. Para asegurar la visibilidad de sombras en la imagen, la captura de las mismas debe hacerse en una franja horaria de 8 a.m. a 11 a.m. y de 2 p.m. a 5 p.m., dependiendo de la época del año y de la latitud. Otra limitación es la captura de imágenes con un cielo cubierto, ya que las sombras se presentan muy tenues. No obstante, este método es una opción aceptable para automatizar la tarea de segmentación ed sombras en imágenes aéreas.
