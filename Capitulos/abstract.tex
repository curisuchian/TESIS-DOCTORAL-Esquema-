El presente trabajo se compone de dos partes. En la primera de ellas se aborda un análisis de distintas plataformas de obtención de imágenes aéreas de áreas de selva nativa, para llevar a cabo el monitoreo de preservación. Se comparan costos y características entre satélites y sistemas aéreos no tripulados (SANT) para diversas areas de selva nativa, tanto reservas de dominio público como privado, con extensiones de pocas hectáreas hasta grandes extensiones. En la segunda parte se analizan distintas formas de procesamiento de imágenes para obtención de información. En particular se abordan tres metodologías para el análisis y procesamiento de imágenes, la detección de sombras por filtrado homomórfico, la segmentación de copas por análisis morfológico y la detección de sombras por índice invariante. Se analizan ventajas y desventajas de cada método, y la viabilidad de aplicar en el monitoreo de las variadas reservas existentes en la provincia de Misiones.\\
\begin{keywords}
    Aerial image, Illumination Invariant Shadow Ratio, Image processing, Remote sensing
\end{keywords}    %Palabras Claves en Español (tres a seis palabras claves separadas por comas. La primera letra de cada palabra clave debe empezar con mayúscula) 

%Aerial image, Illumination Invariant Shadow Ratio, Image processing, Remote sensing