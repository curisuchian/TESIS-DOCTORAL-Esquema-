% Deberá contener el planteo del problema a resolver, los objetivos generales y específicos y una breve explicación de lo que versará la Tesis.

Los bosques nativos presentes en la Tierra brindan una serie de servicios ecosistémicos que benefician a la humanidad, como la captación y filtración de agua, la generación de oxígeno y asimilación de diversos contaminantes, la protección de la biodiversidad, retención de suelo, refugio de fauna silvestre, belleza escénica, entre otros \cite{lin_optimization_2023}. Así, tomar acciones tendientes a su preservación puede ser visto como una cuestión moral, ética, altruista, pero sobre todo, de supervivencia.

Al mismo tiempo, está ampliamente probado el efecto que produce la liberación de carbono en el cambio climático global \cite{noauthor_pdf_nodate, kabir_climate_2023}. En ese contexto, la deforestación y la degradación forestal ocupan el segundo lugar en actividades que contribuyen con la emisión de gases de efecto invernadero, superados solamente por la utilización de combustibles fósiles \cite{chiriaco_deforestation_2024}. Es decir, un bosque que se degrada, no solamente deja de absorber dióxido de carbono del aire, sino que también aporta fuertemente a su liberación.

Afortunadamente, en muchas partes del mundo están proliferando las reservas de bosques nativos. De forma específica, en la provincia de Misiones, Argentina, más de un tercio del territorio se encuentra protegido por algún tipo de legislación nacional, provincial o municipal \cite{noauthor_revista_nodate}. También abundan las reservas privadas interesadas en aportar a la preservación global y que al mismo tiempo proporcionan un rédito económico impulsado por variadas actividades turísticas. Todas estas acciones en pro de la preservación, demandan, como toda actividad consciente, conocer cuan bien se está haciendo el trabajo y disponer de herramientas para el monitoreo y control de bosques nativos se ha vuelto una necesidad mundial.

Ya en 2.007, el Panel Internacional por el Cambio Climático (\textit{International Panel on Climate Change} - IPCC) recomendó el uso de una combinación de datos de observación terrestre (\textit{Earth observation} - EO) e inventarios en campo para estimar el área forestal, el stock de carbono y posibles cambios.  Unos años más tarde, la Convención Marco de las Naciones Unidas sobre el Cambio Climático (\textit{United Nations Framework Convention on Climate Change} - UNFCCC) adoptó un mecanismo que provee incentivos financieros para la reducción de emisiones, llamado REDD+ \cite{pistorius_red_2012}. Para implementarlos, los países son apremiados a establecer un sistema nacional de medición, reporte y verificación (\textit{measurement, reporting and verification} - MRV). Uno de los últimos avances en materia de monitoreo forestal es la misión de la Agencia Espacial Europea llamada Biomass, que tiene entre sus objetivos proveer información de soporte para UNFCCC y REDD++, y será puesta en servicio en el año 2.026.

Cabe resaltar que Argentina completó todos los requisitos para poder participar del mecanismo de financiación REDD+, según lo establecido por la CMNUCC, en su Decisión 1/CP.16, conocido como Acuerdo de Cancún . En los informes técnicos (Referenciar informes técnicos de 2021 y 2023) se señala un abordaje de análisis realizado por especialistas, basados en imágenes satelitales. Argentina cuenta también con un Sistema Integrado de Información Ambiental (https://ciam.ambiente.gob.ar/) dependiente hoy en día de la Subsecretaría de Ambiente, donde se publica la situación de bosques nativos con una periodicidad de actualización anual.

En cuanto a la información de interés en el área forestal, varía desde finos cambios a nivel estructural en altura y dosel, sutiles disrupciones en los servicios ecosistémicos, a grandes pérdidas de biomasa. Estos pueden ocurrir en variadas escalas espaciales y temporales. La floresta degradada puede asumir una cobertura de dosel similar a una floresta intacta, pero tiene menos biomasa, en algunos casos reducida hasta un 75\% \cite{change_report_2006}. Diferentes tipos de floresta responderán de manera diferente a los cambios, con tasas de recuperación variables, dependientes de la localización, tipo, intensidad y extensión de la degradación. Por ello, una sola estrategia de monitoreo no puede ser apropiada a una aplicación a gran escala. En su lugar, un procedimiento específico, adecuado a cada ecosistema es requerido \cite{mitchell_current_2017}. Además. deberían ser considerados aspectos culturales y económicos de cada región. Así, con ese sentido, esta tesis tiene como principal objetivo aportar en diversos aspectos al desarrollo de un sistema de monitoreo de la Selva Misionera.

En la actualidad, cualquier sistema de monitoreo forestal, sea con fines económicos o de preservación, cuenta con dos etapas críticas: la adquisición de imágenes aéreas y la interpretación de dichas imágenes. En esta tesis, se pretende contribuir en estas dos partes, indicando tecnologías y procedimientos que mejor se adecuen a la realidad regional, enmarcada por la escasez de recursos económicos pero motivada por la preservación de la biodiversidad. Así, en cada capítulo que compone esta tesis, se presenta un doble abordaje, considerando siempre plataformas y  procedimientos de adquisición de imágenes y metodologías de procesamiento automatizado de las mismas.
