Recomendaciones para Trabajos Futuros
●	Producción Científica (surgida del trabajo de Tesis)
✔	Publicaciones en Revistas y Capítulos de Libros
✔	Presentaciones a Congresos
●	Proyecto/s de Investigación dentro del/los cual/es se desarrolló la Tesis (si hubiera/n)
●	Beca/s y Subsidio/s con los que se financió la Tesis (si hubieran)
●	Apéndices o Anexos (se reservan para detallar técnicas originales utilizadas o análisis teóricos que impedirían seguir fluidamente el trabajo si se incluyeran en el texto). Las tablas y figuras de los apéndices o anexos deben comenzar otra numeración diferente a la de los capítulos.

Algoritmo morfológico
Preprocesamiento y eliminación de áreas sin sombra
En este script se pretende replicar la experiencia con el algoritmo rollingball del paquete
[baseline] de R.
Inicialmente se deben tener instalados los paquetes necesarios para el procesamiento de
las imágenes.
Este paso puede omitirse si ya se encuentra instalado.
install.packages("ggplot2")
install.packages("imager")
install.packages("mixtools")
install.packages("baseline")
install.packages("Matrix")
install.packages("rgdal")
install.packages("plotly")
install.packages("EcoGenetics")
install.packages("BiocManager")BiocManager::install("EBImage")
Bioconductor version 3.8 (BiocManager 1.30.10), R 3.5.2 (2018-12-20)
Incorporamos las dependencias necesarias
library(mixtools)
library(baseline)
library(ggplot2)
library(imager)
library(Matrix)
library(imager)
library("EBImage")
El primer paso consiste en cargar la imagen en una variable. La representación de la
imagen está en el espacio de color RGB.
#Load and plot the RGB file
data.file <- '../imagenes_forestal/data/original_referencia.jpg'
im <- load.image(data.file)
plot(im)
La imagen con la que se trabaja es un recorte de la imagen del artículo de referencia. Al
no disponer de la versión en color RGB se utiliza  una versión en escala de grises.
La resolución espacial de la fotografía aérea es de 0,5 metro por pixel,
aproximadamente.
#data.file <- '../Captura.png'#recorte <- load.image(data.file)
#plot(recorte)
#data.file <- '../copa_referencia.png'
#recorte <- load.image(data.file)
#plot(recorte)
Una observación en la imagen del recorte de la copa permite estimar el tamaño que
abarca la copa en 80 píxeles. Este dato será usado en los parámetros de los filtro
implementados.
Para procesar la imagen, debe ser convertida al espacio de color  HSL, del cual se
tomará la componente L de luminosidad.
#GPerforms the conversion of the image from RGB to HSL colorspace
im_hsl <- RGBtoHSL(im)
plot(im_hsl)El aspecto de la imagen visualizada en su componente L del espacio HSL es de escala
de grises...
#prints the image L channel from the HSL colorspace. index 3 is L
#plot(channel(im_hsl, 3))
Se extrae la componente L de la imagen HSL y se la almacena en una variable
#Extracts the L component from HSL colorspace image
#im_L <- im_hsl[200:500,200:500,3]
im_L <- im_hsl[,,3]
2.1.2 Estimación de parámetros de distribución bimodal
La función normalmixEM() devuelve una estimación de parámetros de la distribución
de los datos de la componente L de la imagen, considerándola como una distribución
normal bimodal...
#Estimation of bimodal distribution parameters
param <- normalmixEM(as.vector(im_L))
param$munumber of iterations= 114 
0.181090432632824
0.560582899151307
Desplegamos un histograma de los datos del canal L
hist(im_L)
Algoritmo Rolling ball
Se aplica la función baseline() con el método rollingBall y se grafica el resultado. Los
parámetros wm y ws corresponden al ancho de ventana local de minimización y
maximización y de suavizado respectivamente.
#Rolling ball algorithm
#bc.rollingBall <- baseline(im_L, wm=80, ws=80, method='rollingBall')
## Not run: 
#plot(bc.rollingBall)
Para visualizar el efecto del filtro aplicado mediante el algoritmo RollingBall
convertimos los datos corregidos al formato cimg mediante la función as.cimg() y
graficamos.
#corregido <- as.cimg(bc.rollingBall@corrected)
#plot(corregido)Operaciones matemáticas morfológicas
Se utilizan los operadores matemáticos morfológicos de transformaciones top hat y
bottom hat, para mejorar el contraste, basándose en un elemento estructural.
Top hat: es la imagen original en grises menos el resultado de la apertura (erosión
secundada por dilación)
Bottom hat: es la imagen resultante de la cerradura (dilación secundada por erosión)
menos la imagen original en grises Combinando ambos se obtiene el mejoramiento del
contraste sumando a la imagen original el resultado de la transformación top hat y
restando el resultado de la transformación bottom hat: If=I+Ith-Ibh
#Top hat y bottom hat
#mask <- imfill(78,78,val=1)
#top_hat <- as.cimg(im_L) - mopening(as.cimg(im_L),mask)
#bottom_hat <-  mclosing(as.cimg(im_L),mask) - as.cimg(im_L)
#im_filt <- as.cimg(im_L) + top_hat - bottom_hat
#plot(im_filt)
2.1.3 Primer identificación de objetos oscuros
ENTRADA: escala de grises (canal L)
SALIDA: sombras interarbóreas intensificadas
Se lleva a cabo una primera identificación de objetos oscuros, definiéndolos como los
que tienen un valor por debajo de la media en la distribución de grises en brechas
(hallada mediante el algoritmo normalmixEM), y a éstos se los iguala a cero.
entrada3 <- im_L
#values of image that are lower than mean are set to 0
salida3 <- (entrada3>param$mu[1])*entrada3
display(salida3)
hist(entrada3)
hist(salida3)2.1.4 Relleno de sombras en grandes copas de árboles
ENTRADA: escala de grises (canal L)
SALIDA: imagen filtrada
La imagen en escala de grises (canal L) se invierte y se le suma el máximo valor de la
escala de grises. Esto es comparable con una imagen negativa. Se computan dos
imágenes baseline mediante un filtro RollingBall con un radio de tres píxeles. Las
imágenes obtenidas se vuelven a invertir, y los valores máximos de la escala de grises se
usaron para obtener la imagen final suavizada.
entrada4 <- im_L
#inversion of grayscale image and addition of maximum grayscale value
im_L_inv <- entrada4*(-1)+max(entrada4)
#both baselines bline1 and bline2 are computed considering one input as inverted image
and the other input as transposed inverted image
bline1 <- baseline(t(im_L_inv),wm=12, ws=12, method='rollingBall')bline2 <- baseline(im_L_inv,wm=12, ws=12, method='rollingBall')
#smooth image
im_smooth <- pmax(t(bline1@baseline)*(-1),(bline2@baseline)*(-1))
#plot(as.cimg(im_smooth))
salida4 <- im_smooth
hist(entrada4)
hist(salida4)
display(salida4)
2.1.5 Identificar y rellenar huecos en grandes copas de árboles
ENTRADA:
SALIDA:
Se identifican las copas con un diámetro mayor a 15 píxeles, que corresponde a 7,5
metros, mediante una transformación top hat. Para ello se utiliza un elemento
estructurante circular con un diámetro de 15 píxeles. El resultado de esto es una máscara
binaria que contiene solamente las copas de diámetro mayor a 15 píxeles. Los huecos
son rellenados entonces con los valores de escala de grises obtenidos anteriormente.
entrada5 <- salida3
#Top hat
#Structuring element consists in a circular shape of determined radiusradio <- 14 #radius of 7 pixels, corresponding to crown diameter; con 14 se da un mejor
resultado usando la imagen original de referencia
mask <- px.circle(radio)
abertura <- mopening(as.cimg(entrada5),mask,real_mode = FALSE)
t_hat <- as.cimg(entrada5) - abertura
abertura <- abertura>0
display(abertura)
maskara <- abertura[,,1,1]
salida5 <- (maskara*salida4)*(-2)
hist(salida5)
display(salida5)
salida5 <- ((!maskara&entrada5)*entrada5+salida5)
display(salida5)
hist(entrada5)
hist(salida5)

Segunda identificación de objetos oscuros
ENTRADA: salida de la etapa 5
SALIDA: imagen de copas sin sombra interna
Bajo la asunción de que la mayoría de los píxeles sombreados de las copas fueron
removidos, se lleva a cabo una identificación final de píxeles oscuros, los cuales son
definidos como los píxeles de escala de grises que son menores al 99° percentil en la
distribuciones en huecos, y se los iguala a cero.
entrada6 <- salida5
#a normal distribution (n_gaps) is generated, using the parameters that were found with
normalmixEM (eg. the media and standard deviation)
n_gaps <- rnorm(length(entrada6), mean = param$mu[1], sd = param$sigma[1])
noventaynueve <- quantile(n_gaps,.99)
salida6 <- (!(entrada6[,]<noventaynueve))*entrada6
display(salida6)
noventaynueve
hist(entrada6)
hist(salida6)2.1.7 Hallar pequeños huecos en grandes copas
ENTRADA: escala de grises (im_L)
SALIDA: imagen binaria
Las copas grandes poseen píxeles sueltos de sombra que deben ser rellenados para luego
calcular la distancia de los píxeles al borde (o sea los píxeles oscuros). Mediante una
ventana de 7 x 7 píxeles se calcula la ocurrencia de valores distintos de cero entorno a
cada píxel, los cuales poseen una distribución bimodal. Los huecos en las copas se
definen como aquellos que están por encima del 75° percentil. Al final de esta etapa se
identifican tres clases de píxel: los de sombra entre árboles, los no sombreados en las
copas y los aislados de sombras en las copas. Con estas tres clases se compone una
máscara binaria con 0 para píxeles fuera de copas y 1 para los interiores de copas.
entrada7 <- salida6
ti <- proc.time()mat_riz<-cbind(0,0,0,entrada7,0,0,0) #se rellenan tres columnas con ceros por izquierda
y por derecha
mat_riz<-rbind(0,0,0,mat_riz,0,0,0) #se rellenan tres filas con ceros por arriba y por
abajo
MNZ <- entrada7*0 #MNZ es una matriz de la misma dimensión que mat_riz completa
con ceros
for (i in 4:dim(entrada7)[1]+2) { #i es el índice que recorre las columnas
  for (j in 4:dim(entrada7)[2]+2) { #j es el índice que recorre las filas
      a <- i-3
      b <- i+3
      c <- j-3
      d <- j+3
      ventana <- mat_riz[a:b,c:d]
      MNZ[i-2,j-2] <- nnzero(ventana)
  }
 
}
(delta <- proc.time()-ti)
  user  system elapsed 
 24.602   0.001  24.624
En el artículo de referencia se toma un valor de cuantil del 75%; se ha probado con un
valor que se acerca al 94% percentil, ya que ahí el resultado es una matriz nula (todo
negro)
(setentaycinco <- quantile(entrada7,.75))
#hist(MNZ)
#hist(entrada7)
huecos_copas <- (MNZ>setentaycinco)*entrada7
salida7 <- huecos_copas
display(huecos_copas)
mascara7 <- MNZ>setentaycinco
display(mascara7)
display(MNZ)
75%: 0.6078431372549022.1.8 Homogenización de valores de escala de grises en grandes 
copas
ENTRADA: escala de grises (canal L)
SALIDA: imagen binaria
Para homogenizar los valores en grises en grandes copas, se calcula la distancia mínima
entre valores distinto de cero y el valor de cero de la máscara precedente. Todos los
píxeles con distancia mayor a 7 se identifican como grandes árboles, y se rellenan con el
valor de la media de los cuatro valores mayores dentro de una ventana de 7 x 7 píxeles.
entrada8 <- salida7
im_dist <- distmap(MNZ)
#display(im_dist)
mat_riz<-cbind(0,0,0,entrada8,0,0,0) #se rellenan tres columnas con ceros por izquierda
y por derecha
mat_riz<-rbind(0,0,0,mat_riz,0,0,0)
for (i in 4:dim(entrada8)[1]+2) { #i es el índice que recorre las columnas
  for (j in 4:dim(entrada8)[2]+2) { #j es el índice que recorre las filas
      a <- i-3
      b <- i+3      c <- j-3
      d <- j+3
      if (im_dist[i-2,j-2] > 7) {
           ventana <- mat_riz[a:b,c:d]
      media <- 0
      for (n in 1:4) {
          media <- max(ventana)/4 + media
          ventana[which.max(ventana)] <- 0
      }
      entrada8[i-2,j-2] <- media
      }
         
    
  }
 
}
media
salida8 <- entrada8
display(salida8)
hist(salida8)
0.55
#x = readImage(system.file("images", "shapes.png", package="EBImage"))
#display(x)
#dx = distmap(x)#display(dx/10, title='Distance map of x')
2.1.9 Extracción de copas antes de la segmentación
ENTRADA: escala de grises (canal L)
SALIDA: imagen binaria
Las copas con diámetro mayor a 3 metros se extraen mediante un filtro top bottom hat
con elemento estructural cuadrado de 6 x 6 píxeles. A partir de esa imagen
transformada, se aplica un umbral mayor a 0,001° percentil del filtro.
entrada9 <- salida8
#Top hat y bottom hat
mask <- imfill(6,6,val=1)
top_hat <- as.cimg(entrada9) - mopening(as.cimg(entrada9),mask)
bottom_hat <-  mclosing(as.cimg(entrada9),mask) - as.cimg(entrada9)
im_filt <- as.cimg(entrada9) + top_hat - bottom_hat
display(im_filt)
umbral <- 0.1
percentil <- quantile(im_filt,umbral)
salida9 <- (im_filt>percentil)
display(salida9)
#hist(salida9)2.1.10 Delineación de copas individuales
ENTRADA: escala de grises (canal L)
SALIDA: imagen binaria
Se calcula la distancia entre valores cero y distinto de cero, es decir la distancia del
píxel en la copa al borde. Procesando de manera separada por copas o grupos de copas,
calculando las distancias de píxeles a los bordes. Luego se calcula el máximo local en
una ventana cuadrada de la máxima distancia al borde del segmento. Para cada máximo
local se genera una imagen mediante la dilatación entorno a su locación, con un tamaño
que duplique el diámetro.
entrada10 <- salida9
distancia <- distmap(entrada10)
display(distancia)